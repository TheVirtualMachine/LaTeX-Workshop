\documentclass{beamer}

\usepackage{graphicx}
\usepackage{color}
\usepackage{listings}
\usepackage{parskip}
\usepackage{textcomp}

\frenchspacing

\definecolor{commentcolor}{rgb}{0, 0.7, 0.3}
\definecolor{keywordcolor}{rgb}{0.6, 0, 0.5}
\definecolor{stringcolor}{rgb}{0, 0, 0.9}

\lstset {%
	language=[LaTeX]TeX,
	basicstyle=\footnotesize\ttfamily,
	breaklines=true,
	numbers=left,
	tabsize=4,
	commentstyle=\color{commentcolor},
	keywordstyle=\color{keywordcolor},
	stringstyle=\color{stringcolor},
	columns=flexible,
	morekeywords={\therefore, \because},
}

\title{Introduction to \LaTeX{}}
\subtitle{A \LaTeX{} Presentation}
\author{Vincent Macri\\JJ Marr}
\date{2017}

\usetheme{Rochester}
\usecolortheme{orchid}
\beamertemplatenavigationsymbolsempty
\usefonttheme[onlymath]{serif}
\logo{\rmfamily\normalsize\LaTeX{}}

\AtBeginSection[] {%
	\begin{frame}
	\frametitle{Lesson Progress}
	\tableofcontents[currentsection]
	\end{frame}
}

\begin{document}
	\frame{\titlepage}
	\section{Introduction}
	\begin{frame}
		\frametitle{\secname}
		\framesubtitle{Welcome to \LaTeX{}}
		\begin{itemize}[<+->]
			\item Lamport \TeX{}
			\item Written as \LaTeX{} or LaTeX.
			\item Pronounced lay-tek
		\end{itemize}
	\end{frame}
	\begin{frame}
		\frametitle{\secname}
		\framesubtitle{Why we use \LaTeX{}}
		\LaTeX{} is a very useful skill if you plan on doing:
		\pause
		\begin{itemize}
			\item Math
			\item Computer science
			\item Engineering
			\item Physics
			\item Chemistry
			\item Publishing
		\end{itemize}
		\pause
		Since \LaTeX{} is so versatile, once you get used to it, you will find that you can do almost anything in it.
	\end{frame}
	\begin{frame}
		\frametitle{\secname}
		\framesubtitle{Learning \LaTeX{}}
		The hardest part about learning \LaTeX{} is getting started.
		\pause

		So congratulations! You're done the hard part!
		\pause

		When using \LaTeX{}, it is important to remember that it makes correct design choices 99\% of the time. Accept what it does, as it does it for a reason.
		\pause

		For example, \LaTeX{} uses very large margins by default. It may look odd, but it actually sets the margins to make lines the size that studies have shown is best for readability (60--70 characters). This is easy to change of course.
	\end{frame}
	\begin{frame}
		\frametitle{\secname}
		\framesubtitle{Basic syntax}
		\begin{description}[<+->]
			\item[Spacing] Spaces in \LaTeX{} don't do much. \LaTeX{} treats 20 spaces the same as it treats 1 space.
			\item[Inline math] \lstinline{\$1 + 2    =3\$} $\rightarrow 1 + 2 = 3$\\ In math mode, you can use 0 or 20 spaces, \LaTeX{} will output the same thing. Do whatever you find easiest to write and edit.
			\item[Comments] Comment \LaTeX{} code with the \% symbol. Write \textbackslash\% to actually use the \% symbol.
			\item[Commands] \LaTeX{} uses commands to perform quick actions. They are formatted as \textbackslash command\{parameter\}. For some commands, the \{\} and parameter are optional.
			\item[Environments] \LaTeX{} uses environments to create things too complex for commands. They start with \textbackslash begin\{environment\} and end with \textbackslash end\{environment\}
		\end{description}
	\end{frame}
	\begin{frame}
		\frametitle{\secname}
		\framesubtitle{Installing}
		Installing \LaTeX{} is easy, but there are lots of options for installing it.
		
		Listed below are the easiest solutions for the main desktop operating systems. Some of these work on all operating systems, but certain ones are easier with certain operating systems.
		\begin{description}
			\item[GNU/Linux and BSD] Install \TeX{} Live from your distro's repo
			\item[macOS] Mac\TeX{} is supposed to be good
			\item[Windows] MikTeX is a distribution that comes with many programs for editing and compiling \LaTeX{}, such as TexWorks.
		\end{description}
		If you have difficulty installing \LaTeX{}, you can try an online editor such as overleaf.com

		You can also check https://www.latex-project.org/get/ if you are not satisfied with any of these options.
	\end{frame}
	\begin{frame}
		\frametitle{\secname}
		\framesubtitle{Let's start}
		In this presentation, we will show how to write a \LaTeX{} document that proves the quadratic formula.
	\end{frame}
	\section{Preamble}
	\begin{frame}
		\frametitle{\secname}
		\framesubtitle{The beginning}
		Our document is split into two sections.

		The \alert{preamble} and the \alert{main document}.
	\end{frame}
	\begin{frame}
		\frametitle{\secname}
		\framesubtitle{The preamble}
		Your preamble is where you tell \LaTeX{} about what is going to be in your document.

		Information is the preamble includes information such as:
		\begin{itemize}[<+->]
			\item Document class
			\item Packages
			\item Document title
			\item Author
			\item Date (defaults to date when document was compiled)
			\item Miscellaneous information about your document
		\end{itemize}
	\end{frame}
	\begin{frame}
		\frametitle{\secname}
		\framesubtitle{Example}
		Let's code our preamble.
		\lstinputlisting[linerange=1-12]{QuadraticProof.tex}
	\end{frame}
	\section{Setup}
	\begin{frame}[fragile]
		\frametitle{\secname}
		\framesubtitle{Subtitle}
		\LaTeX{} uses \alert{sections} to split up parts of the document.

		In school assignments, you will usually have each question be its own section. If a question has more than one ``subquestion'', you can create a ``subsection'' as well.
		\begin{example}
			\begin{lstlisting}[numbers=none, gobble=16]
				\section{Prove that $1 + 1 = 2$}\\
				\subsection{Extend the previous proof to $1+2 = 3$}
			\end{lstlisting}
		\end{example}
	\end{frame}
	\begin{frame}[fragile]
		\frametitle{\secname}
		\framesubtitle{Title page and table of contents}
		The \lstinline{\maketitle} command is used to generate a title page.
		\begin{example}
			\begin{lstlisting}[numbers=none, gobble=16]
				\maketitle
			\end{lstlisting}
		\end{example}
		
		The \lstinline{\tableofcontents} command will generate a neat table of contents out of your sections.
		\begin{example}
			\begin{lstlisting}[numbers=none, gobble=16]
				\tableofcontents
			\end{lstlisting}
		\end{example}
		If your items aren't showing up in your table of contents, try recompiling.
	\end{frame}
	\section{The Document}
	\begin{frame}[fragile]
		\frametitle{\secname}
		\framesubtitle{Let's actually start}
		Time to actually start our document!

		Let's explain how we prove the quadratic formula using our words before we derive it.
		\begin{example}
			\begin{lstlisting}[numbers=none, gobble=16]
				\section{Prove the quadratic formula}
				We will prove this by deriving it from $ax^2 + bx + c = 0$.

				This is fairly trivial.
			\end{lstlisting}
		\end{example}
	\end{frame}
	\begin{frame}[fragile]
		\frametitle{\secname}
		\framesubtitle{Doing math}
		Most of the math we do right now is just using formulas and solving for things.

		The \alert{align} environment is useful for this. We put an asterisk into the name of the environment to get rid of equation numbering.
		\begin{example}
			\begin{lstlisting}[numbers=none, gobble=16]
				\begin{align*}
				\end{align*}
			\end{lstlisting}
		\end{example}
	\end{frame}
	\begin{frame}[fragile]
		\frametitle{\secname}
		\framesubtitle{Math symbols}
		Here is a list of useful \LaTeX{} math symbols:
		\begin{description}	
			\item[$\therefore$] \lstinline{$\therefore$}
			\item[$\because$] \lstinline{$\because$}
			\item[$\approx$] \lstinline{$\approx$}
			\item[$\rightarrow$] \lstinline{$\rightarrow$}
			\item[$\neq$] \lstinline{$\neq$}
			\item[$\pm$] \lstinline{$\pm$}
			\item[$\times$] \lstinline{$\times$}
			\item[$\cdot$] \lstinline{$\cdot$}
			\item[$\infty$] \lstinline{$\infty$}
		\end{description}	
	\end{frame}
\end{document}
